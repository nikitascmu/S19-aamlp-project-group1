\documentclass[twoside,10.5pt]{article}
\usepackage{jmlr2e}
%\usepackage{multicol}
%\usepackage[landscape]{geometry}
\usepackage{hyperref}
\usepackage{endnotes}
\let\footnote=\endnote
\renewcommand{\notesname}{Endnotes}
\newcommand{\dataset}{{\cal D}}
\newcommand{\fracpartial}[2]{\frac{\partial #1}{\partial  #2}}
\ShortHeadings{95-845: AAMLP Proposal}{Gangwar, Rost and Setia}
\firstpageno{1}

\begin{document}

\title{Heinz 95-845: Project Proposal}

\author{\name Mridul Gangwar \email mgangwar@andrew.cmu.edu \\
       \addr Heinz College of Information Systems and Public Policy\\
       Carnegie Mellon University, Pittsburgh, PA, United States \
       \AND
       \name Lauren Rost \email lrost@andrew.cmu.edu \\
       \addr Heinz College of Information Systems and Public Policy\\
       Carnegie Mellon University, Pittsburgh, PA, United States \
       \AND
       \name Nikita Setia \email nikitas@andrew.cmu.edu \\
       \addr Heinz College of Information Systems and Public Policy\\
       Carnegie Mellon University, Pittsburgh, PA, United States}
       
\maketitle
%\begin{multicols}{2}
%\section{Project Details} \label{details}
\vspace*{5px}
\section{Introduction}
Opioid abuse has been an emerging public health issue, and was declared a Public Health Emergency in 2017\footnote{\cite{HHS}}. The Centers for Disease Control and Prevention (CDC) estimates that an average of 130 people die in the U.S. every day from opioid overdose\footnote{\cite{Wonder}}. Many nationally-funded public health institutions like the CDC and NIH, have established goals, public health initiatives, and funding schemas to combat this epidemic\footnote{\cite{CADCA}}\footnote{\cite{CDC_OO}}\footnote{\cite{NationalInstitute}}. Research has been dedicated towards addressing this major public health issue as well, however opioid abuse and overdose remains a point of public health interest.   

\section{Methods}
\subsection{Pre-processing and Data Cleaning}
\subsection{Cohort}
*Table 1
Inclusion and Exclusion criteria 
Imputation methods
Machine learning methods (MLR, regularized MLR, random forest, boosting, gradient boosting, k-means clustering)
Cross-validation


\section{Results}
AUROC, Figures 

\section{Discussion}
Significance of results
Limitations 

\newpage
\theendnotes
\bibliographystyle{ieeetr}
\bibliography{proposal_bibliography.bib}

\end{document} 
