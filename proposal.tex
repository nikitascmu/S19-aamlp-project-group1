\documentclass[twoside,11pt]{article}
\usepackage{jmlr2e}
\newcommand{\dataset}{{\cal D}}
\newcommand{\fracpartial}[2]{\frac{\partial #1}{\partial  #2}}
\ShortHeadings{95-845: AAMLP Proposal}{Gangawar, Rost and Setia}
\firstpageno{1}

\begin{document}

\title{Heinz 95-845: Project Proposal}

\author{\name Mridul Gangawar \email mgangawar@andrew.cmu.edu \\
       \addr Heinz College of Information Systems and Public Policy\\
       Carnegie Mellon University\\
       Pittsburgh, PA, United States \
       \AND
       \name Lauren Rost \email lrost@andrew.cmu.edu \\
       \addr Heinz College of Information Systems and Public Policy\\
       Carnegie Mellon University\\
       Pittsburgh, PA, United States \
       \AND
       \name Nikita Setia \email nikitas@andrew.cmu.edu \\
       \addr Heinz College of Information Systems and Public Policy\\
       Carnegie Mellon University\\
       Pittsburgh, PA, United States}
\maketitle


\section{Introduction}



\section{Project Overview} \label{details}

\subsection{Proposed analysis and Likely Outcomes}

\subsection{Importance of Proposed Analysis}

\subsection{Contribution to Existing Work}

% Provide references, \emph{e.g.}, see: \cite{cite1}.

\subsection{Data Description}

% Where applicable, please also define Y outcome(s), U treatment, V covariates, and W population.

\subsection{Evaluation Measures}

\subsection{Study Design, Pre-Processing and Machine Learning Methodology}

% Justify that the analysis is of appropriate size for a course project.

\subsection{Possible Study Limitations}

\subsection{Potential Future Use of Analytic Pipeline}

\bibliography{sample.bib}
%\appendix
%\section*{Appendix A.}
%Some more details about those methods, so we can actually reproduce them.

\end{document}
